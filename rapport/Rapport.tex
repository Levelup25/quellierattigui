\documentclass[a4paper,12pt]{article}
\usepackage{times}
\usepackage[francais]{babel}
\usepackage[utf8x]{inputenc}

\usepackage[T1]{fontenc}
\usepackage{hyperref}
\usepackage[shortlabels]{enumitem}
\setlength{\parindent}{0pt}
\setlength{\parskip}{\baselineskip}

\usepackage{amsmath}
\usepackage{amssymb}

\usepackage{graphicx}
\graphicspath{ {res/} }

\usepackage{pdfpages}
\usepackage{pdflscape}

\usepackage{listings}
\usepackage{longtable}
\lstset{literate=
{é}{{\'e}}1
{è}{{\`e}}1
{ê}{{\^e}}1
{à}{{\`a}}1
{â}{{\^a}}1
}
\lstset{language=C++,
                basicstyle=\footnotesize,
                keywordstyle=\footnotesize\color{blue},
                otherkeywords={override,nullptr}
}
\definecolor{orange}{rgb}{0.8,0.4,0.0}
\definecolor{darkblue}{rgb}{0.0,0.0,0.6}
\definecolor{cyan}{rgb}{0.0,0.6,0.6}
\lstdefinelanguage{JSON}
{
  basicstyle=\normalsize,
  columns=fullflexible,
  showstringspaces=false,
  commentstyle=\color{gray}\upshape,
  morestring=[b]",
  morestring=[s]{>}{<},
  morecomment=[s]{<?}{?>},
  stringstyle=\color{orange},
  identifierstyle=\color{darkblue},
  keywordstyle=\color{blue},
  morekeywords={string,number,array,object}% list your attributes here
}

\sloppy

\setlength{\topmargin}{0cm}
\setlength{\headsep}{0.in}
\setlength{\headheight}{0.in}
\setlength{\evensidemargin}{0cm}
\setlength{\oddsidemargin}{-1cm}
\textwidth 18cm
\textheight 25cm

\begin{document}

\thispagestyle{empty}

\begin{titlepage}

\vspace*{2cm}

\begin{center}\textbf{\Huge Projet }\end{center}{\Large \par}

\begin{center}\textbf{\large Attigui Youness - Quellier Louis}\end{center}{\large \par}

\vspace{2cm}

  \begin{figure}[ht]
        \centering
        \includegraphics[width=\textwidth]{dofus.png}
        \caption{Exemple du jeu}
    \end{figure}


\clearpage

{\small
\tableofcontents
}

\end{titlepage}

\clearpage
\section {Présentation Générale}

\subsection{Archétype}
    Pour ce projet nous nous inspirons du gameplay du jeu \href{http://www.jeuxvideo.com/jeux/pc/00013232-dofus.htm}{"Dofus"}. Il s'agit d'un jeu de stratégie au tour par tour où il est possible de se déplacer de carte en carte pour combattre des ennemis et gagner de l'équipement et de l'expérience permettant d'améliorer son personnage et d'avancer plus loin dans le monde.

\subsection{Règles du jeu}
    Vous incarnez un héros auto-proclamé qui vient de fonder sa guilde d'aventurier dans le but de libérer le royaume de l'emprise du roi démon.
    
    % description aventure (début, victoire, défaite)
    L'aventure commence avec une arme basique au choix, combiné avec un élément de prédilection. Ces armes pourront être trouvées ou achetées et gagneront des niveaux selon le nombres d'ennemis vaincus, débloquant ainsi des bonus de statistiques ainsi que des nouvelles actions, pouvant également être améliorées. Ce niveau est inhérent à l'arme et sera alors conservé lors du transfert vers un autre personnage. 

	% description stats des personnages / ennemis
    Tous les personnages du jeu possèdent en plus de leur arme un ensemble de caractéristiques propres à chacun :
    \begin{itemize}[noitemsep,topsep=0pt]
        \item pv : point de vie (s'ils tombent à 0, le personnage meurt et lâche son matériel sur la case sur laquelle il se trouvait)
        \item pa : point d'action (permettent d'utiliser les actions associées aux armes)
        \item pm : point de mouvement (permettent de se déplacer)
        \item éléments : eau/terre/feu/air
    \end{itemize}

    % description map
    Les différents éléments (l'eau, la terre, le feu et l'air) font partie intégrante de ce monde et sont au coeur des stratégies de combat. Des bonus et malus de type élémentaire sont appliqués selon l'attribut du joueur, de son arme et de la case du terrain sur lequel il est placé. C'est à vous d'utiliser le terrain et vos compétences à votre avantage.
    
    % mouvement map et mission
    Au cours de l'histoire, d'autres aventuriers rejoindront votre guilde afin de vous aider dans votre tâche et vous serez engagés pour effectuer des missions variées afin de gagner de l'or pour acheter de nouvelles armes ou recruter de nouveaux mercenaires.
    
    En tant que guilde indépendante vous êtes libres de vous déplacer à votre guise dans le royaume pour accomplir les missions dans l'ordre que vous souhaitez. Néanmoins, dans ce monde il n'existe pas d'organisme régissant les missions, celles-ci vous sont confiées lorsque vous arrivez sur les lieux.
    
    Par moments vous devrez défendre une ville contre une attaque adverse, parfois vous devrez protéger un voyageur surpris par une embuscade ennemie ou souhaitant une escorte, d'autres fois vous devrez détruire une base fortifiée mais le plus souvent vous devrez simplement éliminer tous les adversaires présents dans une zone.
    
    % description terrain
    Un combat se déroule sur un des terrains de la carte du monde. Chaque terrain est composé d'un nombre fixe de cases. Chaque case est associé à un élément (eau, terre, feu, air, neutre). Le déploiement des personnages s'effectue au début du combat sur une zones prédéterminée dépendant de la direction d'arrivée, la disposition à l'intérieur de cette zone étant libre.
    
    % réputation
    Au cours de votre aventure votre guilde acquiert une certaine notoriété dans le royaume, celle-ci peut aussi bien être positive que négative, influant sur vos interactions avec les différents personnages. Au fil de vos rencontres, vous pourrez refuser certaines missions pour diverses raisons mais cela fera diminuer votre réputation. Le degré de réussite ou d'échec des missions acceptées modifiera également cette réputation.
    
    Si votre guilde a trop mauvaise réputation, vous serez exécutés en tant que criminel et l'aventure s'arrête là. Au contraire une très bonne réputation vous donne le privilège d'accéder à l'armurerie de Vulc contenant des armes surpuissantes mais également excessivement chères...
    
    % fin de l'aventure et recommencer aventure
     L'aventure se termine après avoir vaincu l'un des 4 boss élémentaires, vous permettant débloquer son arme associée. Vous pouvez cependant perdre la partie si tous vos personnages meurent ou via le système de réputation. Les armes rencontrées au cours de l'aventure seront stockées dans votre armurerie personnelle, vous permettant de choisir une nouvelle arme de départ en recommençant une aventure. La progression d'expérience est perdue mais l'or est gardé entre chaque partie.
    
\subsection{Ressources}

    \begin{figure}[ht]
        \centering
        \includegraphics[width=\textwidth]{persos.png}
        \caption{Tileset personnages}
    \end{figure}
    
    \begin{figure}[ht]
        \centering
        \includegraphics{monsters-32x32.png}
        \caption{Exemple tileset ennemis}
    \end{figure}
    
    \begin{figure}[ht]
        \centering
        \includegraphics[width=\textwidth]{field.png}
        \caption{Exemple tileset terrain}
    \end{figure}


\clearpage
\section{Description et conception des états}

\subsection{Description des états}
Un état de jeu est formé de la carte du monde possédant des cases d'un certain élément, repérées par ses coordonnées dans le monde, pouvant éventuellement posséder un obstacle infranchissable inamovible ou un personnage mobile. Ce personnage fait partie d'une équipe possédant un inventaire contenant des objets, ceux-ci étant principalement des armes pouvant être équipées individuellement et possédant diverses capacités et caractéristiques. L'équipe du joueur peut combattre et obtenir des quêtes fournissant des récompenses lors de leur accomplissement.
Le détail des attributs et opérations des différentes classes est présenté ci-dessous.

\subsubsection{State}
\begin{itemize}
\item L'état possède un pointeur sur la carte du monde qui contient toutes les information lié au terrain et à la position des personnages
\item L'heure correspondant au nombre de fronts d'horloge, ainsi que la vitesse à laquelle le jeu est actualisé permet d'associer un temps à chaque état du jeu
\item Un booléen indique si le joueur est en train de combattre ou non, et un autre indique si l'inventaire est ouvert
\end{itemize}

\subsubsection{World}
\begin{itemize}
\item Le monde possède un nombre prédéfini de lignes et de colonnes pour la carte
\item Il possède aussi un tableau à 2 dimensions de toutes les cellules du jeu
\item La liste des personnages est aussi stocké
\end{itemize}

\subsubsection{Cell}
\begin{itemize}
\item La case doit contenir un élément choisi parmis une énumération et peut contenir un obstacle infranchissable par les personnages
\end{itemize}

\subsubsection{Character}
\begin{itemize}
\item Les coordonnées du personnage
\item Le personnage a un nom et appartient à une race qui sera utile par la suite pour les quêtes notamment
\item Il possède également des attributs utiles au combat lui permettant de se déplacer, d'attaquer et de pouvoir encaisser des attaques
\item Chaque personnage possède enfin une arme lui permettant d'effectuer diverses actions
\end{itemize}

\subsubsection{Team}
\begin{itemize}
\item L'équipe comprend l'ensemble des personnages aptes au combat. Seul un combattant représente l'équipe hors combat lors du déplacement sur la carte du monde, et seul un nombre limité de personnages peuvent combattre simultanément
\item L'équipe stocke également l'inventaire des objets récupérés, pouvant être affectées aux différents personnages de l'équipe
\end{itemize}

\subsubsection{Inventory}
\begin{itemize}
\item L'inventaire comprend différents objets, entre autres les armes affectées aux combattants et celles en réserve
\end{itemize}

\subsubsection{Weapon}
\begin{itemize}
\item L'arme peut posséder un nom
\item L'arme possède des capacités permettant d'attaquer ou de se défendre
\item Elle possède également un élément de prédilection répercuté sur son porteur
\end{itemize}

\subsubsection{Ability}
\begin{itemize}
\item Les habilités peuvent avoir un nom, coûtent un certain nombre de points d'action, infligent un certain nombre de dégâts (ou de soin en cas de valeur négative) et possèdent éventuellement un temps de recharge de la capacité
\item Elles possèdent un élément répercuté sur l'attaque en elle même. Elle sera souvent du même élément que l'arme, mais pourront parfois différer
\item L'attaque possède une portée effective, et une zone d'effet dépendant du type d'arme utilisé
\item Une réduction de dégât peut être envisagée pour les attaques de zone afin de moins affecter les personnages éloignés du point d'impact
\end{itemize}

\subsubsection{Fight}
\begin{itemize}
\item Le combat enregistre le nombre de tours passés afin de définir la prochaine équipe à jouer
\item Il possède les différentes équipes se battant afin que chaque membre de l'équipe puisse se déployer avant de commencer
\item Il possède également des quêtes permettant de gagner des récompenses en cas de réussite
\end{itemize}

\subsubsection{MainQuest}
\begin{itemize}
\item Les quêtes principales s'obtiennent avant d'entrer en phase de combat en effectuant certaines actions et peuvent se décliner en divers objectifs à remplir à travers plusieurs combats à travers ses classes filles
\end{itemize}


\subsection{Conception Logiciel}
Le diagramme des classes pour les états est présenté en Figure 5 : 
\begin{itemize}
\item Les classes en vert contiennent les information globales sur le déroulement du jeu et permettent de déterminer l'état actuel du jeu
\item Les classes en jaune caractérisent les personnages et le terrain dans lequel ils évoluent
\item Les classes en bleu déterminent les actions possibles par les différents personnages
\item Les classes en rouge permettent d'implémenter les phases de combat ainsi que les quêtes, permettant d'obtenir du matériel ou de nouveaux membres d'équipe.
\end{itemize}

\begin{landscape}
    \begin{figure}[p]
    \includegraphics[width=0.9\paperheight]{state.png}
        \caption{\label{uml:render}Diagramme des classes de rendu.} 
    \end{figure}
\end{landscape}

\clearpage
\section{Règles de changement d'états et moteur de jeu}

\subsection{Règles}

\clearpage
\subsection{Conception logiciel}


%\begin{landscape}
%\begin{figure}[p]
%\includegraphics[width=0.9\paperheight]{engine.pdf}
%\caption{\label{uml:engine}Diagramme des classes de moteur de jeu.} 
%\end{figure}
%\end{landscape}


\section{Intelligence Artificielle}

\subsection{Stratégies}

\clearpage
\subsection{Conception logiciel}


%\begin{landscape}
%\begin{figure}[p]
%\includegraphics[width=0.9\paperheight]{ai.pdf}
%\caption{\label{uml:ai}Diagramme des classes d'intelligence artificielle.} 
%\end{figure}
%\end{landscape}


\section{Modularisation}
\label{sec:module}

\subsection{Organisation des modules}

\clearpage
\subsection{Conception logiciel}


%
%\begin{landscape}
%\begin{figure}[p]
%\includegraphics[width=0.9\paperheight]{module.pdf}
%\caption{\label{uml:module}Diagramme des classes pour la modularisation.} 
%\end{figure}
%\end{landscape}

\end{document}

